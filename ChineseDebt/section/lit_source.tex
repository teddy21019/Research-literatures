\begin{frame}
    \frametitle{中國債務研究困難}

    \begin{itemize}
        \item 過去不清楚中國對重債窮國提供金援的數量與程度
        \item 傳統會透過民營機構的借貸來觀測與評比主權債務,但中國對外借款皆透過官方
        \item 透過政府、政策性銀行、國有商業銀行的資料,中國不會回報至傳統機構
        \item 中國非巴黎俱樂部成員
    \end{itemize}

\end{frame}


\begin{frame}
    \frametitle{新資料庫}

    \begin{itemize}
        \item \citet*{Horn-Reinhart-Trebesch-21} 整理出 4900 筆借貸\&贈與資料,包含年分、國家、機構、支付形式等。
        \begin{itemize}
            \item 1949 -- 2017
            \item  2017 中國成為最大官方債權人(超越 IMF、WB)。
            \item 超過50\%對外放款未被記錄在傳統資料庫 -- ``Hidden debts''。
        \end{itemize}
        \item \citet*{HPRT-23} 進一步整理中國人民銀行\footnotemark{}在其他國家的央行上的貨幣互換 (swap line)的資料庫。
        \begin{itemize}
            \item Swap line 成為最主要金融援助工具
        \end{itemize}
    \end{itemize}

    \footnotetext{中國人民銀行即為中國央行,以下簡稱 PBOC。}

\end{frame}