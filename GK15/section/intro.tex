\begin{frame}
    \frametitle{Banking Distress and the Economy}

    \begin{enumerate}
        \item Gertler and Kiyotaki (2011)
        \begin{itemize}
            \item Depletion of capital
            \item Ability to raise fund
            \item Cost of bank credit
            \item Slows the economy
        \end{itemize}
        \item Diamond and Dybvig (1983)
        \begin{itemize}
            \item Liquidity mismatch
            \item Inefficient asset liquidation 
            \item Possibility of bank run
        \end{itemize}
    \end{enumerate}

    Most models capture one, but lack the other
    \begin{itemize}
        \item Financial acceleration effect, but no bank run 
        \item Bank run, but not connected to fundamentals
    \end{itemize}

\end{frame}

\begin{frame}
    \frametitle{The 2008 Crisis}
    Ben Bernanke (2010) and Gorton(2010) 
    \begin{itemize}
        \item Weakening financial position led to classical runs 
        \item Usually on shadow banking sectors
    \end{itemize}

    Historical fact: 
    \begin{enumerate}
        \item Slow run --- Mar 2008, Bear Stearns. Creditors are reluctant to deposit
        \item Fast run --- Sep 2008, Lehman Brothers. Collapse of the entire shadow banking system.
    \end{enumerate}

\end{frame}

\begin{frame}
    \frametitle{This paper}

    \begin{enumerate}
        \item Balance sheet condition affect both cost of bank credit and possibility of bank run 
        \item Anticipation of run can affect asset price.
    \end{enumerate}
    
    Approach closer to Cole and Kehoe (2000) on \emph{self-fulfilling debt crisis}. 
    Therefore no \emph{sequential service constraint}. 

\end{frame}

