\begin{frame}
    \frametitle{Finite expected lifetime}
    \begin{exampleblock}{100 percent equity financing}
        To the extent bankers may face financial market frictions, 
        they will attempt to save their way out of the financing constraint by
        accumulating retained earnings in order to move toward 100 percent equity financing.  
    \end{exampleblock}

    To limit this possibility --- probability $\sigma$ of surviving, 
    hence expected lifetime $\frac{1}{1-\sigma}$ 
    \begin{enumerate}
        \item Meaning?
        \item How does limited lifetime help solve this? In reality? 
    \end{enumerate}
    
\end{frame}

\begin{frame}
    \frametitle{The Balance Sheet}

    \begin{itemize}
        \item Net worth (Assets - liabilities) for surviving bankers
        \begin{equation}
            n_t = (Z_t + Q_t) k^b_{t-1} - R_{t}d_{t-1}
        \end{equation}
        \item Net worth for new bankers --- get initial endowment
        \begin{equation}
            n_t^\text{new banker} = w_b
        \end{equation}
        \item Exiting bankers consume all their net worth 
        \begin{equation}
            c^b_t = n_t
        \end{equation}
        \item Bank finance assets holdings with new deposit + net worth (retained earnings)
        \begin{equation}
            Q_t k^b_t = d_t + n_t
        \end{equation}
        \item Banks choose $\{d_t, k^b_t\}$ each period to maximize its franchise value.
    \end{itemize}

\end{frame}

\begin{frame}
    \frametitle{Agency Problem}
    \begin{itemize}
        \item A banker could play honest --- hold depositors' asset until realization in $t+1$, and pay deposit 
        \item Banker would also want to divert some assets for personal use
        \item Assume can sell $\theta \in (0,1)$ of assets secretly 
    \end{itemize}

    \vfill 
    
    \begin{alertblock}{Risk of doing so? }
        Depositors can force a bankruptcy in the next period. \\
        The gain from diverting fund can not exceed the franchise value.
    \end{alertblock}
    That is, 
    \begin{equation}
        \theta Q_t k^b_t \le V_t
    \end{equation}
\end{frame}

\begin{frame}
    \frametitle{Franchise Value and Incentive Constraint}

    \begin{block}{Franchise Value}
        Present discount value fo future payouts from operating honestly.
        \begin{equation}
            V_t = E_t [\beta(1-\sigma)n_{t+1} +\beta\sigma V_{t+1}] 
        \end{equation} 
        Higher franchise value reduces excessive risk taking strategy by banks (See Demsetz, Saidenberg and Strahan, 1996)       
    \end{block}

    The franchise value is constant return to scale, 
    so the bank's optimization problem is reduced to maximizing the Tobin's Q
    \begin{block}{Tobin's Q}
        In this context, franchise value is its market value, 
        and the replacement cost is defined by its net worth
        \begin{equation*}
            \Psi_t \equiv V_t/n_t
        \end{equation*}
    \end{block}

\end{frame}

\begin{frame}
    \frametitle{Tobin's Q}

    Dividing the franchise value by its net worth, we get 
    \begin{equation*}
        \frac{V_t}{n_t} = E_t\left[
            \beta(1-\sigma)\frac{n_{t+1}}{n_t} +\beta\sigma \frac{V_{t+1}}{n_{t+1}}\frac{n_{t+1}}{n_t}
            \right]
    \end{equation*}
    \begin{align}
        \frac{n_{t+1}}{n_t} =& \frac{(Z_{t+1} + Q_{t+1}) k^b_{t} - R_{t+1}d_{t}}{n_t} \nonumber \\
        =& \underbrace{\frac{Z_{t+1} + Q_{t+1}}{Q_{t}}}_{\equiv R^b_{t+1}} \underbrace{\frac{Q_t k^b_t}{n_t}}_{\phi_t} - R_{t+1} \frac{d_t}{n_t} \nonumber \\
        =& (R^b_{t+1} - R_{t+1})\phi_t + R_{t+1}
    \end{align}
\end{frame}

\begin{frame}
    \frametitle{Bank's problem --- max Tobin's Q}
    The bank's problem become choosing leverage 
    \begin{equation}
        \psi_t = \max_{\phi_t} \{\mu_t\phi_t + \nu_t\}
    \end{equation}

    The incentive constraint 
    \begin{equation}
        \theta Q_t k^b_t \le V_t \implies \theta \phi_t \le \psi_t = \mu_t\phi_t + \nu_t
    \end{equation}
    \begin{align}
        \mu_t &= E_t [\beta\Omega_{t+1}(R^b_{t+1} - R_{t+1})] & \text{\tiny Excess M.V of assets over deposit}\\
        \nu_t &= E_t [\beta\Omega_{t+1}R_{t+1}] & \text{\tiny M.C of deposit} \\
        \Omega_{t+1} &= (1-\sigma)\times 1 + \sigma \times \psi_{t+1} & \text{\tiny M.V of net worth} \nonumber
    \end{align}

\end{frame}

\begin{frame}
    \frametitle{IC binding}
    The IC binds when $\theta \phi_t = \psi_t = \mu_t\phi_t + \nu_t$, while $\theta_t \in (0,1)$, 
    we must satisfy $0 < \mu_t < \theta $ and 
    \begin{equation}
        \phi_t = \frac{\psi_t}{\theta} = \frac{\nu_t}{\theta - \mu_t}.
    \end{equation}

    When IC is binding
    \begin{itemize}
        \item Portfolio size is balanced by (cost of losing) franchise value
        \item Fluctuation in net worth induce fluctuation in bank lending
        \item \emph{Moreover, cannot negative net worth. } Otherwise, incentive constraint that ensures the bankers will not divert is violated. How?
    \end{itemize}
\end{frame}