\begin{frame}
    \frametitle{Bank Run in CBDC}

    Many have discussed about the possibility of run if CBDC is issues in a closed economy.
    \begin{enumerate}[<+->]
        \item Direct deposit in central bank that is fully insured
        \item Can be interest bearing as a monetary policy
        \item Competition with commercial banks, causing disintermediation by crowding out deposits \footnote{See Fernandez-Villaverde, J., D. Sanches, L. Schilling, and H. Uhlig (2021) for a model}
        \item Financial instability (Gertler and Kiyotaki, 2015)
    \end{enumerate}
    \pause
    \vfill 
    The impact could be mitigated by proper regulations and limits on the design of CBDC.

\end{frame}

\begin{frame}
    \frametitle{Bank Run Under Foreign CBDC}

    However, if CBDC is cross-border, domestic country might be out of means.

    \begin{enumerate}[<+->]
        \item Currency substitution in high-inflation country (Calvo, 1992)
        \item CBDC is digital, easy to access using mobile devices, causing faster substitution rate 
        \item Commercial banks and domestic central bank both losses deposits, causing even more severe run and financial instability.
        \item Currency substitution could ultimately impair monetary policy (Ferrari et al. 2022)
    \end{enumerate}

\end{frame}


\begin{frame}
    \frametitle{This Paper}

    Extends DD to cross-border CBDC.

    Chooses the following foreign CBDC design
    \begin{itemize}
        \item Account-base \emph{v.s. token}
        \item Retail \emph{v.s. Wholesale}
        \item Cross border
        \item Interest-bearing
    \end{itemize}
    \pause
    The domestic country has no CBDC technology, while foreign country issues cross-border CBDC that could cause capital outflows. 

\end{frame}

