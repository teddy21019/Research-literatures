\begin{frame}
    \frametitle{Consumers}

    Period 0~: \\
    Endowed with 1 unit of goods. 
    Can decide to save domestically or abroad. \\
    \vspace{2em}
    Period 1~: \\
    Consumers find out whether they are patient or not.

    \begin{equation}
        U(c_1, c_2) = \begin{cases}
            u(c_1) & \text{with prob. } \lambda \\
            u(c_2) & \text{with prob. } 1-\lambda
        \end{cases}
    \end{equation}
    If impatient, consume $c_1$ at t=1
    \\
    \vspace{2em}

    Period 2~: \\
    Patient consumers consumer $c_2$
\end{frame}

\begin{frame}
    \frametitle{Techonology}

    \begin{table}
        \begin{tabular}{cccc}
            Type & Period 0 & Period 1 & Period 2 \\
            \hline
            Short-term & 1 & 1 & \\
            
            Long-term & 1 & $l<1$ & $R>1$\\
            \hline
        \end{tabular}
    \end{table}

    Directly take the result that consumers will want to save in bank(foreign or domestic) to pool the risk.

\end{frame}

\begin{frame}
    \frametitle{Domestic Commercial Banks}

    \begin{itemize}
        \item Domestic Commercial Banks (DB) offers a demand deposit contract $(c_1, c_2)$. 
        \item DB decide to invest $y\in(0,1)$ in the short-term technology.
    \end{itemize}

    \begin{block}{Sequential Service Constraint}
        DB pays $c_1$ to depositors until all resources are exhausted.        
    \end{block}

\end{frame}

\begin{frame}[allowframebreaks]
    \frametitle{DB's problem}

    \begin{equation}
        \max_{c_1, c_2, y, {\color{red} l} \in \mathbb{R}^3_+} 
        \lambda u(c_1) + (1-\lambda)u(c_2)
    \end{equation}

    s.t. 
    \begin{subequations}
        \begin{align}
            0 &\le y \le 1 \label{eq:3a} \\
            \lambda c_1 &\le ry + (1-y)l \label{eq:3b} \\
            (1-\lambda) c_2 &\le R{\color{red} (1-l)}(1-y) + ry + (1-y)l - \lambda c_1 \label{eq:3c} \\
            c_1 &\le c_2 \label{eq:3d}
        \end{align} \label{eq:db-constraint}
    \end{subequations}
    \begin{alertblock}{Abuse of notation}
        Previously denote $l$ as liquidation price, but now denote $l$ as the proportion of long-term used for early liquidation.        
    \end{alertblock}

    \framebreak 

    Key intuitions 
    \begin{itemize}
        \item Assume bankers are Bertrand competition $\implies$ forced to maximize E.U of consumers 
        \item~\ref{eq:3b} --- might use liquidated long-term to pay $c_1$
        \item~\ref{eq:3c} --- Non-liquidated long-term plus after return is yield plus the leftover from period 1
        \item~\ref{eq:3d} --- Incentive compatibility constraint, patient consumers don't pretend to be impatient
    \end{itemize}

\end{frame}

\begin{frame}
    \frametitle{Foreign CBDC-issuing Central Bank}

    \begin{itemize}
        \item Assume perfect credible, i.e.\ no runs
        \item Justification: Fernandez-Villaverden et al. (2021) point out, certain punishment (for early withdrawal) and treatment (with patient depositors) ensures no run on CB is a DSE
        \item Foreign CBDCs are open to foreigners
        \item Constract --- $(c^*_1, c^*_2)\in \mathbb{R}^2_+$
    \end{itemize}
\end{frame}

\begin{frame}
    \frametitle{Consumers' Problem}

    Consumer invest in the highest ex-ante utility
    \begin{enumerate}
        \item Pick the contract, denote $d_i\in\{0,1\}$. $d_1=0$ means save in domestic.
        \item If utility ties, fraction $f\in[0,1]$ of consumers pick the foreign contract
    \end{enumerate}

    Consumers also decide when to withdraw their funds, denote $w_i\in \{1,2\}$. Note that $w_i=w_i (w_{-i})$
    \vfill
    Capital account constraint --- Total investment in foreign asset must not exceed $k$
    \begin{itemize}
        \item Exogenous ceiling
        \item Regulation restriction
    \end{itemize}
\end{frame}