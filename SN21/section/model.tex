\begin{frame}
    \frametitle{Timeline in one cycle}
    \small
    \begin{columns}[t]
        \begin{column}{0.33\textwidth}
            Subperiod 0 
            \rule{\textwidth}{1pt}
            Agent draw random preference $U_i \sim [0,1]$
            
            \vspace{1em}
            Agent endowed with 1

            \vspace{1em}
            $\bar{U}_i<0.5$ - Impatient, invest in liquid asset  
    
            \vspace{1em}
            $\bar{U}_i>0.5$ - Patient, invest in illiquid asset
           
            \vspace{1em}
            Alternatively - Become depositor
        \end{column}

        \begin{column}{0.33\textwidth}
            Subperiod 1 
            \rule{\textwidth}{1pt}
            Liquidity shock

            $\rho_{i} = \bar{U}_i + (-1)^{b_i}\frac{\epsilon_i}{2}$
            
            \vspace{1em}
            Nondepositor - Search to trade asset. 
            
            \vspace{1em}
            Depositors - Choose to withdraw $r_1$ or not
            
            \vspace{2em}
            Liquid asset - Repay 1, Illiquid asset - Repay $r$
            
            \vspace{1em}
            Bank - Face withdrawal, decide to default
        \end{column}

        \begin{column}{0.33\textwidth}
            Subperiod 2 
            \rule{\textwidth}{1pt}
            Holding illiquid asset - Receive $R$
            
            \vspace{2em}
            Depositors - Receive $r_2$
        \end{column}
    \end{columns}
\end{frame}

\begin{frame}
    \frametitle{Preference Shock}
    There is an initial preference shock at t=0 (subperiod is denoted as t)
    \begin{equation*}
        U_i \sim \operatorname{Unif}[0,1]
    \end{equation*}

    Denote its realization as $\bar{U}_i$
    
    A new preference shock in t=1 
    \begin{equation*}
        \rho_{i} = \bar{U}_i + (-1)^{b_i}\frac{\epsilon_i}{2}
    \end{equation*}
    $b_i \sim \operatorname{Bernoulli}(0.5)$,  $\epsilon_i\sim \operatorname{Unif}[0,1]$

    For both $U$ and $\rho$, $>0.5$ represents patient, vise versa.
\end{frame}

\begin{frame}
    \frametitle{Rate of return}
    \begin{table}
        \centering

        \begin{tabular}[h]{r c c}
            Type & t=1 & t=2 \\
            \hline 
            Liquid Asset & $1$ & \\
            Illiquid Asset & $r$ & $R$ \\
            Deposit & $r_1$ & $r_2$ \\
            \hline            
        \end{tabular}
    \end{table}

    \begin{equation*}
        r<1<r_1<r_2<R
    \end{equation*}
    The rate of returns are publicly known to everybody.

\end{frame}

\begin{frame}
    \frametitle{Bargain}
    Bargain happens between asset holders that have inconsistent intertemporal preference. 
    \begin{itemize}
        \item Impatient in t=0, patient in t=1 (Positive preference shock)
        \item Patient in t=0, impatient in t=1 (Negative preference shock)
    \end{itemize}
    \vfill
    Might not find a partner in his social network $v$. The Moore neighborhood in this case.
\end{frame}

\begin{frame}
    \frametitle{Becoming a bank}

    \begin{itemize}
        \item Decision is made in t=0, depending on the impatient agents in his social network $v$.
        \item Unknown proportion of impatient agents $w \in \{0, \frac{1}{9}, \dots,  1\}$
    \end{itemize}

    Become a banker if per capita present value must provide is less than endowment:
    \begin{equation*}
        f(w_i) = w_i r_1 + (1-w_i)\frac{r_2}{R} \le 1
    \end{equation*}
    or
    \begin{equation*}
        w_i \le \frac{R-r_2}{R r_1 - r_2}
    \end{equation*}
\end{frame}

\begin{frame}
    \frametitle{Not honoring / discouraging of creation}

    For each value $Q \in (1, r_1)$, there exist $\omega \in [w^*, 1]$ such that $f(\omega)=Q$.
    Where $f(w^*) = 1$. 
    \\[2em]
    There are realizations of $w$ that discourage the creating or incentive to default.
\end{frame}

\begin{frame}
    \frametitle{Investment decision of Bank}

    Within the per capita present value the bank must provide
    \begin{equation*}
        f(w_i) = w_i r_1 + (1-w_i)\frac{r_2}{R}
    \end{equation*}
    
    \begin{itemize}
        \item $x_i = (1-w_i)\frac{r_2}{R}$ - Investment in illiquid assets.
        \item $y_i = w_i r_1$ - Investment in liquid assets.
        \item $1 - x_i - y_i$ - Added to reserve
    \end{itemize}
    \pause
    Intuition 
    \\[1em]
    The bank ensures $y_i=w_i r_1$ of liquid assets to provide withdrawal in first period, 
    and ensures $R x_i = (1-w_i)r_2$ of illiquid assets to provide withdrawal in the second period.
    The rest are kep as reserves.
\end{frame}

\begin{frame}
    \frametitle{Opening a bank account}
    \begin{block}{Becoming a client}
        $T(v, pyf):$ If the agent evaluates that it is advantageous,
        he opens an account in a bank in the immediate neighborhood; 
        if there is none in this condition, he becomes a client of the same bank of one of his neighbors. 
    \end{block}
    \begin{itemize}
        \item $v$ --- Agent's social network
        \item $pyf$ --- Result of the comparison of payoffs.
    \end{itemize}

    $pyf$ is determined by a learning rule.
\end{frame}

\begin{frame}
    \frametitle{Decision of becoming a client}

    \begin{itemize}
        \item Follows Grasselli and Ismail (2013)
        \item Agents have memory of 5 cycles.
        \item Memory information has three states.
        \begin{itemize}
            \item[N] --- If the \emph{budget constraint} remains unchanged after the shock 
            \item[B] --- There was a change but no partner was found.
            \item[G] --- There was a change and someone to bargain with was found
        \end{itemize}
        \item Total of 7 predictors
    \end{itemize}
\end{frame}

\begin{frame}
    \frametitle{Predictors I}
    \begin{enumerate}
        \item k will be the same as k-1
        \item k will be the same as t-2
        \item ... t-3
        \item ... t-4
        \item ... t-5
        \item k will be equal to mode of last 3 previous cycle
        \item k will be equal to mode of last 5 previous cycle
    \end{enumerate}

    Each predictor maps to a forecast of one of the 3 states, 
    (I) denote $\Theta = [\theta_1, \theta_2, \dots, \theta_7]$, where $\theta_i \in \{N,B,G\}$
\end{frame}

\begin{frame}

    \begin{quote}
        In the decision to become a bank customer, the agent can map the return of each predictor to a situation
        in which he deposits or not his cash in bank, obtaining, respectively, the vectors $\underset{1\times 7}{\Pi_d}$ and $\Pi_n$
    \end{quote}

    \pause

    \begin{alertblock}{How is the return calculated?}
        Note that $\bar{U}_i$ is initialized during decision
        \begin{enumerate}
            \item Do agents have to consider the probability of being patient when deposit??
            \item Will all elements of $\Pi_d$ be the same?
            \item Do agents consider present value on t=1?
        \end{enumerate}
    \end{alertblock}
\end{frame}

\begin{frame}
    \frametitle{Desision of Becoming a Client (Con'd)}
    Decision 

    \begin{equation*}
        A^* = \argmax_{A\in\{d, n\}}  \Pi_A \cdot \Phi
    \end{equation*}
    $\underset{1 \times 7}{\Phi}$ is the weight, called ``force'', of the predictor vectors.
    \vfill
    That is, the agent decides to become a client of its neighbor's bank, or the same as his neighbor, if $\Pi_d \cdot \Phi > \Pi_n \cdot \Phi$

    \vfill
    
    The bank is chosen as its neighbor's bank.
\end{frame}

\begin{frame}
    \frametitle{Law of Motion for the Weight}

    For each of the predictor, $+1$ to the corresponding weight if correctly forecasted, and $-1$ if the not.

    \begin{equation*}
        \phi_{j, t+1} = (-1)^{\I{\theta_{j,t}=\hat{\theta}_t}} + \phi_{j, t}
    \end{equation*}
    
    Where $\hat{\theta}_t \in \{N,B,G\}$ is the true realization state at time t. 

\end{frame}

\begin{frame}
    \frametitle{Withdrawal}

    \pause
    \begin{block}{Sequential service constraint}
        Closer clients withdraw first. \\ 
        Random assign as the tie-breaking rule.
    \end{block}
    \pause
    \vfill
    Withdrawal period (normal case)
    \begin{enumerate}
        \item Impatient depositors ($\rho < 0.5$) --- withdraw at t=1, get $r_1$
        \item Patient depositors ($\rho < 0.5$) --- withdraw at t=2, get $r_2$
    \end{enumerate}

\end{frame}

\begin{frame}
    \frametitle{Imitation Rule}
    The key of bank run is the allowance of patient clients to imitate the decision of neighbors (in its social network)

    \begin{block}{Imitation Rule}
        If $\rho>\frac{1}{2}$ but more than $n$ neighbors in his social network $v$ intend to withdraw in now (t=1), then the agent withdraws.
    \end{block}

\end{frame}

\begin{frame}
    \frametitle{Bank's Behavior}
    The bank pays $r_1$ in t=1, and pays $r_2$ in t=2, the rest is saved as reserves.
    \vfill
    The expected proportion of impatient agents in the next cycle follows an adaptive rule 
    \begin{equation*}
        w^k_i = w^{k-1}_i + \alpha (\bar{w}_i - w^{k-1}_i)
    \end{equation*}
    Where $\bar{w}_i$ is the actual proportion of impatient clients.
    
\end{frame}

\begin{frame}
    \frametitle{Fail of a Bank}
    As mentioned before, closest bank clients withdraw first.

    \begin{block}{Order of assests used to pay}
        \begin{enumerate}
            \item Liquid assets
            \item Reserve 
            \item Illiquid assets
        \end{enumerate}
    \end{block}

    If the bank exhausts all its resources, the remaining clients receive nothing and they break link with the bank.

    \begin{block}{Fail}
        \# of clients $\le 5$, it fails. The remain clients are released.
    \end{block}

\end{frame}