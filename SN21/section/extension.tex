\begin{frame}
    \frametitle{Allowing Wealth Accumulation}

    In previous baseline model, each agent returns to the unit endowment.
    \pause
    \vfill 
    If we allow depositors to accumulate wealth: 
    \begin{enumerate}
        \item Bank reserve large, harder for bank run to occur.
        \item Withdrawal amount increases, easier for bank run to occur.
    \end{enumerate}

    The extension checks if this scenario is possible.

\end{frame}

\begin{frame}
    \frametitle{Modification --- Wealth}

    Some agents might never withdraw in t=1
    \begin{itemize}
        \item Wealth follows a geometric progression growth 
    \end{itemize}
    \vfill 
    \begin{itemize}
        \item $\omega_i$ --- available wealth 
        \item 1 unit of endowment at each t=0 
        \item Agents allocate $\omega_i$ in asset market or deposit
        \item \color{red}{Spends $\omega_i$ each cycle --- (Sort of) Hand to mouth setting}
    \end{itemize}
\end{frame}

\begin{frame}
    \frametitle{Allocation decision of $\omega_i$}

    Agents not clients of bank 
    \begin{itemize}
        \item Impatient --- Spent all in t=1. $\omega_{i, t+1}=0$
        \item Patient --- No spending in t=1, receive $R\omega_i$ in t=2, spend $\omega_i$
    \end{itemize}

    Agents who are clients of bank
    \begin{itemize}
        \item Impatient --- received $r_1 \omega_i$ in t=1, spend $\omega_i$ in t=1
        \item Patient --- No spending in t=1, receive $r_2 \omega_i$ in t=2, spend $\omega_i$
    \end{itemize}
\end{frame}

\begin{frame}
    \frametitle{Banks' allocation}

    \emph{The authors did not mentioned the modification on the formation and allocation of banks.}

    The orders of offering money is the same 
    \begin{enumerate}
        \item Liquid assests
        \item Reserves 
        \item Illiquid assets
    \end{enumerate}

    If the bank has not enough to pay, some clients receive nothing, and the bank fails.
\end{frame}

\begin{frame}
    \frametitle{Thoughts}

    \begin{itemize}[<+->]
        \item A great start to explore dynamic bank runs.
        \item Bank run emerges from simple imitation rule, which considers only limited knowledge from the environment(neighbors), not a global information.
        \item Endogenously selection of becoming clients of bank.
    \end{itemize}
\end{frame}

\begin{frame}
    \frametitle{Improvements and criticisms}
    \pause
    Criticisms
    \begin{enumerate}[<+->]

        \item Consumption smoothing
        \item Expected value of a forecast not described clearly
    \end{enumerate}
    \vfill
    Adjustment and Improvements
    \begin{enumerate}
        \item Existing large banks. Endogenous bank formation is not necessary in most applications. 
        \item Exogenous bank-depositor network --- Spacial network is not enough.
        \item Consumption and preference shock should be endogenous and follow a transition process.
    \end{enumerate}

\end{frame}